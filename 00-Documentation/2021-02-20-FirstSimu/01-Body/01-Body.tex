\section*{Simulation of nuSTORM production straight}
A rudimentary simulation of nuSTORM has been created in python.
The simulation is based on the design presented in~\cite{Ahdida:2020whw}.
To initiate consideration of detector systems and sensitivities the
following parameters have been adopted for the storage ring, the
production straight and the muon-beam optics:
\begin{itemize}
  \item Total ircumference: 616\,m
  \item Length of production straight: 180\,m
  \item Stored muon momentum ($p_\mu$) range: $1 \le p_\mu \le 6$\,GeV
  \item Momentum acceptance: $\pm 15$\%
    \begin{itemize}
      \item Simulate a parabolic momentum distribution with limits
        $\pm 15$\%
    \end{itemize}
  \item Transverse acceptance: 1\,$\pi$\,mm\,rad
  \item Transverse beta function (in both transverse coordinates: 25000\,mm
\end{itemize}
The transverse beta function is taken as “representative” of the
production straight.
It is assumed that $\alpha=0$, and that the emittance, $\epsilon$,
(acceptance) and beta are momentum independent. 
The width of the transverse phase space is obtained using:
\begin{eqnarray}
  x_i        & = & \sqrt{\epsilon \beta} \quad {\rm ; \quad and} \nonumber \\
  x_i^\prime  & = & \sqrt{\frac{\epsilon}{\beta}} \quad ;  \nonumber
\end{eqnarray}
where $x_i$ refers to both transverse coordinates ($x$ and $y$).

\section*{Neutrino flux interface specification}
\textbf{\color{red} To be drafted.}
